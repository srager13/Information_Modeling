\section{Information Model}
\label{sec:info_model}

There are a number of agents that get factoids distributed to them.  Each of these factoids has about 1-3 pieces of information that can contribute to a solution, which itself is made up of 4 parts- who, what, where, and when.  Each contributes to the QoI by contributing to one or more parts of this solution.  For example, a factoid could be like "The Lion was seen in Psiland," which contributes to both the who and the where.  All of the agents are working toward their own solution, using the factoids they receive while also sharing them with their neighbor agents.  

For simplicity in evaluation, we can reduce the problem to a small number of categories, even just one, like who, for example.  We can also make some simplifications of the QoI contribution of each factoid to the solution part(s).  We can make a requirement that some number X of factoids need to be collected to get to the correct solution.  Then, we can expand it to a probabilistic problem in which the parts of each factoid add to a confidence probability in a particular solution.

We will assume a scenario in which $N$ pieces of intelligence, called factoids, are supplied to an agent.  Each of the $N$ factoids are uniquely identified by $i = \{1,..., N\}$, representing the order in which they are introduced to the agent.  We will represent the agent's ability to recall factoid $i$ with the random variable $R_i$ where $P(R_i = 1) = p_{r_i}$ and $0$ otherwise.  The value $p_{r_i}$ will be explained in Section \ref{sec:ser_pos_effect}.  

%\begin{eqnarray*}
%  f_{R_i} &=&
%    \left\{\begin{array}{ll}
%    p_{r_i} & \mbox{    } r_i = 1 \\
%    1-p_{r_i} & \mbox{    } r_i = 0 
%    \end{array}\right.
%\end{eqnarray*}